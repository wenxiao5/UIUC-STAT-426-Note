%!TEX program = xelatex
\documentclass[11pt,a4paper]{article}
\usepackage[utf8]{inputenc}
\usepackage[T1]{fontenc}
\usepackage{authblk}
\usepackage{ctex}
\usepackage{tikz}
\usepackage{pgfplots}
\usepackage{verbatim}
\usepackage{amsfonts}
\usepackage{amsmath}
\usepackage{amsthm}
\usepackage{enumerate}
\usepackage{indentfirst}
\usepackage{amssymb}
\setlength{\parindent}{0pt}
\usetikzlibrary{shapes,snakes}
\newcommand{\argmax}{\operatornamewithlimits{argmax}}
\newcommand{\argmin}{\operatornamewithlimits{argmin}}
\DeclareMathOperator{\col}{col}
\usepackage{booktabs}
\newtheorem{theorem}{Theorem}
\newtheorem{note}{Note}
\newtheorem{definition}{Definition}
\newtheorem{proposition}{Proposition}
\newtheorem{lemma}{Lemma}
\newtheorem{example}{Example}
\newtheorem{corollary}{Corollary}
\usepackage{graphicx}
\usepackage{geometry}
\usepackage{hyperref}
\newcommand{\code}{	exttt}
\geometry{a4paper,scale=0.8}
\title{STAT 426}
\author[*]{Wenxiao Yang}
\affil[*]{Department of Mathematics, University of Illinois at Urbana-Champaign}
\date{2022}
\begin{document}
\maketitle
\tableofcontents
\newpage

\section{Basic of Categorical Data}
\subsection{Variable Measurement}
\begin{center}\begin{figure}[htbp]
    \centering
    \includegraphics[scale=0.5]{p1.png}
    \caption{Variable Type}
    \label{}
\end{figure}\end{center}
\begin{enumerate}[a)]
    \item Nominal: Categories do not have a natural order. Ex. blood type, gender.
    \item Ordinal: Categories have a natural order. Ex. low/middle/high education level
    \item Interval: There is a numerical distance (difference between two different values is meaningful) between any two values.
    Ex. blood pressure level, 100 blood pressure doesn't mean the double degree of 50 pressure.
    \item Ratio: An interval variable where ratios are valid (presence of absolute zero, i.e. zero is meaningful). Ex. weight, 4g is double degree of 2g.
\end{enumerate}


\end{document}